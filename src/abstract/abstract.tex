\section*{Abstract}
\addcontentsline{toc}{section}{Abstract}
This report provides a description of the road marking recognition project.

The Duckietown project uses a light-sensitive road marking recognition algorithm as a baseline. It is fast and consistent in constant light conditions. However when 
the lightning changes (different weather, time or even just one broken lamp in the room) this algorithm should be adjusted by changing its config file. 
It is a pretty suboptimal solution, which makes developers of the Duckietown project rebuild their solutions each time the light conditions are changed. 
Assuming that the Duckietown project imitates real autonomous driving cars and the idea of the smart city, rebuilding the solution for autopilot because of the 
change of the daytime doesn't look like a good imitation of the real system. So it was decided to create a new algorithm for detecting road markups, based on a deep learning 
(DL) algorithm. This approach makes the algorithm light-independent.
The report describes in detail the mathematical foundations of the algorithms used in the project, as well as a description of the overall design and implementation. 
In addition, collecting and marking up the dataset for DL is described in this report.
As a result, the algorithm was created and the dataset was published. This algorithm was tested in Joint Advanced Student School (JASS) 2024 by JetBrains and was pretty 
successful.
